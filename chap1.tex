\chapter*{ABC 365 - E 解説}
    \section*{問題文}
    \url{https://atcoder.jp/contests/abc365/tasks/abc365_e}

    \section*{前提知識}
    高校数学, 排他的論理和, 累積和, Java もしくは Python の文法

    \section*{解説}
    [方針] 本問題は XOR の性質を利用する問題であり, XOR においては, 各ビットを独立して計算できることを利用する.
    すなわち, 本問題の数式を $2$ 進数表記で書き表すことから考えると良い.\\

    $A_i$ を $2$ 進数表記すると,
    $$ A_i = \sum_{k=1}^{30} 2^{k - 1} B_{i, k} $$
    となる.
    ここで, $B_{i, k}$ は $0$ か $1$ である.
    $A_i$ の最大値は問題文の制約より $10^8$ なので $10^8 < 2^{30}$ であることを利用すると, $A_i$ の 2進数表記は $30$ 桁以内であることが分かる.
    そのため $k$ の上限値は $30$ としている.

    $2$ 進数表記の具体例として, $A_1 = 13$ の場合, $A_1 = 1101_{(2)}$ となるので, $B_{1, 1} = 1, B_{1, 2} = 0, B_{1, 3} = 1, B_{1, 4} = 1$ とすると,
    $$A_1 = 2^0 B_{1, 1} + 2^1 B_{1, 2} + 2^2 B_{1, 3} + 2^3 B_{1, 4}$$
    となる($k \ge 5$ は明らかに $B_{1, k} = 0$ なので省略している).

    本問題は, 次の式を計算する問題であった.
    $$\sum_{i = 1}^{N - 1}\sum_{j = i + 1}^N (A_i \oplus A_{i + 1} \oplus \cdots \oplus A_j)$$
    先ほどの $2$ 進数表記を用いると, XOR がビットごとに独立して計算できることから,
    $$\sum_{k = 1}^{30} 2^{k - 1} \left(\sum_{i = 1}^{N - 1}\sum_{j = i + 1}^N (B_{i, k} \oplus B_{i + 1, k} \oplus \cdots \oplus B_{j, k})\right)$$
    となる.
    これ以降では, 大きな $\left(\right)$ で囲まれた部分の計算について考える.

    累積和のように, 累積 XOR を導入すると, $B_{i, k} \oplus B_{i + 1, k} \oplus \cdots \oplus B_{j, k}$ を簡潔に書くことができ, 計算の高速化も可能である.
    これが実際にそうであることを確認していく.

    累積和で計算のための配列を用意し, 初期値を $0$ とするように, 累積 XOR の配列 $C$ を用意し, 初期値を $0$ とする.
    形式的には次のようになる.
    $$C_0 = 0, C_i = B_1 \oplus B_2 \oplus \cdots \oplus B_i$$
    もし, $C$ が累積和の場合は形式的には次のようになる.
    $$C_0 = 0, C_i = B_1 + B_2 + \cdots + B_i$$
    このように, $+$ の部分を $\oplus$ に置き換えることで, 累積 XOR を考えることができる.

    累積和では, $C_j - C_{i - 1} = B_i + B_{i + 1} + \cdots + B_j$ であった.
    累積 XOR においても同様のことが成り立つ.
    すなわち, $C_j \oplus C_{i - 1} = B_i \oplus B_{i + 1} \oplus \cdots \oplus B_j$ である.
    このことが成り立つことを示す.\\

    まず, $C_j$ = $B_1 \oplus B_2 \oplus \cdots \oplus B_{i - 1} \oplus B_i \oplus \cdots \oplus B_j$ が言える.

    また, $C_{i - 1}$ = $B_1 \oplus B_2 \oplus \cdots \oplus B_{i - 1}$ が言える.

    $C_j \oplus C_{i - 1} = ({\color{red}{B_1 \oplus B_2 \oplus \cdots \oplus B_{i - 1}}} \oplus B_i \oplus \cdots \oplus B_j) \oplus ({\color{red}{B_1 \oplus B_2 \oplus \cdots \oplus B_{i - 1}}})$

    XOR においては, $A \oplus A = 0$ であることから, 赤色の部分が消える.

    すなわち, $C_j \oplus C_{i - 1} = B_i \oplus B_{i + 1} \oplus \cdots \oplus B_j$ である.
    よって, 成り立つことが示された.\\

    引き算は交換法則が成り立たないが, XOR は交換法則が成り立つので, $C_j \oplus C_{i - 1} = C_{i - 1} \oplus C_j$ である.

    $$\sum_{i = 1}^{N - 1}\sum_{j = i + 1}^N (B_{i, k} \oplus B_{i + 1, k} \oplus \cdots \oplus B_{j, k})$$
    先述した上記の式に累積 XOR を導入すると次のように式変形できる.
    以下では, 数式を簡潔にするためにちょっとした計算を行う.
    ここはあまり本質的ではないので, 興味がない場合は読み飛ばしても良い.
    $$\sum_{i = 1}^{N - 1}\sum_{j = i + 1}^N (C_{i - 1, k} \oplus C_{j, k})$$
    一番左のシグマを $i = 1$ から $i = 0$ にすると,
    $$\sum_{i = 0}^{N - 2}\sum_{j = i + 2}^N (C_{i, k} \oplus C_{j, k})$$
    となる.
    左から二番目のシグマを $j = i + 2$ から $j = i + 1$ にすると,
    $$\sum_{i = 0}^{N - 2}\sum_{j = i + 1}^N (C_{i, k} \oplus C_{j, k}) - \sum_{i = 0}^{N - 2} (C_{i, k} \oplus C_{i + 1, k})$$
    となる.
    一番左のシグマを $N - 2$ から $N - 1$ にすると,
    $$\sum_{i = 0}^{N - 1}\sum_{j = i + 1}^N (C_{i, k} \oplus C_{j, k}) - \sum_{i = 0}^{N - 2} (C_{i, k} \oplus C_{i + 1, k}) - (C_{N - 1, k} \oplus C_{N, k})$$
    初項以外は, 次のように一つの項にまとめることができる.
    $$\sum_{i = 0}^{N - 1}\sum_{j = i + 1}^N (C_{i, k} \oplus C_{j, k}) - \sum_{i = 0}^{N - 1} (C_{i, k} \oplus C_{i + 1, k})$$
    ここで, $C_{i, k} \oplus C_{i + 1, k} = B_{i + 1, k}$ であるから,
    $$\sum_{i = 0}^{N - 1}\sum_{j = i + 1}^N (C_{i, k} \oplus C_{j, k}) - \sum_{i = 0}^{N - 1} B_{i + 1, k}$$
    となる.
    一番右のシグマを $i = 0$ から $i = 1$ にすると,
    $$\sum_{i = 0}^{N - 1}\sum_{j = i + 1}^N (C_{i, k} \oplus C_{j, k}) - \sum_{i = 1}^{N} B_{i, k}$$
    となる.

    この式は, 大きな $\left(\right)$ で囲まれた部分の計算であったので, その外側を付け加えると,
    $$\sum_{k = 1}^{30} 2^{k - 1} \left(\sum_{i = 0}^{N - 1}\sum_{j = i + 1}^N (C_{i, k} \oplus C_{j, k}) - \sum_{i = 1}^{N} B_{i, k}\right)$$
    となる.
    分配法則を用いて, この式を展開すると,
    $$\sum_{k = 1}^{30} 2^{k - 1} \left(\sum_{i = 0}^{N - 1}\sum_{j = i + 1}^N (C_{i, k} \oplus C_{j, k})\right) - \sum_{k = 1}^{30} 2^{k - 1} \left(\sum_{i = 1}^{N} B_{i, k}\right)$$
    となる.
    まず, 右側の式について考える.
    $$\sum_{k = 1}^{30} 2^{k - 1} \left(\sum_{i = 1}^{N} B_{i, k}\right)$$
    これは, $A_i$ の総和を $2$ 進数表記で考えたものであるから, 
    $$\sum_{i = 1}^{N} A_i = \sum_{k = 1}^{30} 2^{k - 1} \left(\sum_{i = 1}^{N} B_{i, k}\right)$$
    である.
    したがって, 式は次のようになる.
    $$\sum_{k = 1}^{30} 2^{k - 1} \left(\sum_{i = 0}^{N - 1}\sum_{j = i + 1}^N (C_{i, k} \oplus C_{j, k})\right) - \sum_{i = 1}^{N} A_i$$
    次に, 右側の式について考える.
    実はこの式は真面目に計算するよりも, 素早く求める方法がある.
    $C$ は累積 XOR であるから, $C_i \in \{0, 1\}$ である.
    それから, 以下の部分を見ると, $i < j$ であるような $(i, j)$ のペアを考えていることが分かる.
    $$\sum_{i = 0}^{N - 1}\sum_{j = i + 1}^N$$
    $0 \oplus 0 = 0, 0 \oplus 1 = 1, 1 \oplus 0 = 1, 1 \oplus 1 = 0$ であるので, $C_{i, k} \oplus C_{j, k}$ の値が $1$ となるには $C_{i, k} \neq C_{j, k}$ である必要がある.
    つまり, 以下の式を計算することは, $C_{i, j} \neq C_{j, k}$ であるペアの個数を求めることに等しい.
    そのようなペアの個数を求めることは, $C$ 中にある $0$ の個数と $1$ の個数の積に等しいことも分かる.
    $$\sum_{i = 0}^{N - 1}\sum_{j = i + 1}^N (C_{i, k} \oplus C_{j, k})$$
    よって, $C$ 中にある $0$ の個数と $1$ の個数を求めることで, 右側の式を求めることができ, 次に $A_i$ の総和を引くことで, 答えを求めることができる.
    あとはこれをプログラムすれば良い.

    \section*{実装:Python}
    ソースコード\ref{python_ans}に Python による解法を示す.
    このコードは, 下記URLにある私の提出コードから確認可能である.

    \url{https://atcoder.jp/contests/abc365/submissions/56340290}
\begin{lstlisting}[caption=Pythonのコード, label=python_ans]
n = int(input())
a = list(map(int, input().split()))

bit_SIZE = 30

# 1 つ目の式
def f(bit_idx):
    # cnt[0] := 0 の個数
    # cnt[1] := 1 の個数
    # 初期値 0 についてもカウントするので cnt[0] = 1 としている
    cnt = [1, 0]

    c_last_val = 0
    for i in range(n):
        if a[i] >> bit_idx & 1:
            c_last_val ^= 1
        else:
            c_last_val ^= 0
        cnt[c_last_val] += 1
    return cnt[0] * cnt[1] << bit_idx

ans = 0
for bit_idx in range(bit_SIZE + 1):
    ans += f(bit_idx) # 1 つ目の式
ans -= sum(a) # 2 つ目の式

print(ans)
\end{lstlisting}

    \section*{実装:Java}
    ソースコード\ref{java_ans}に Java による解法を示す.
    このコードは, 下記URLにある私の提出コードから確認可能である.

    \url{https://atcoder.jp/contests/abc365/submissions/56340269}
\begin{lstlisting}[caption=Javaのコード, label=java_ans]
import java.util.*;

public class Main {
    public static void main(String[] args) {
        Scanner scanner = new Scanner(System.in);
        int n = scanner.nextInt();
        int[] a = new int[n];
        for (int i = 0; i < n; i++) {
            a[i] = scanner.nextInt();
        }
        int bit_SIZE = 30;

        long ans = 0;
        for (int bit_idx = 0; bit_idx <= bit_SIZE; bit_idx++) {
            ans += f(bit_idx, a, n); // 1 つ目の式
        }
        long sum_a = 0;
        for (int i = 0; i < n; i++) {
            sum_a += a[i];
        }
        ans -= sum_a; // 2 つ目の式

        System.out.println(ans);
    }

    // 1 つ目の式
    public static long f(int bit_idx, int a[], int n) {
        // cnt[0] := 0 の個数
        // cnt[1] := 1 の個数
        // 初期値 0 についてもカウントするので cnt[0] = 1 としている
        int[] cnt = {1, 0};

        int c_last_val = 0;
        for (int i = 0; i < n; i++) {
            if ((a[i] >> bit_idx & 1) != 0) {
                c_last_val ^= 1;
            } else {
                c_last_val ^= 0;
            }
            cnt[c_last_val]++;
        }
        return (long) cnt[0] * cnt[1] << bit_idx;
    }
}
\end{lstlisting}
